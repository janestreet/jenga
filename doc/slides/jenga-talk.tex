\documentclass{beamer}

\usepackage{helvet}
\usepackage{textpos}
\usepackage{setspace}

\useinnertheme{circles}

\definecolor{jswhite}{rgb}{0.9,0.9,0.9}
\definecolor{jsyellow}{rgb}{0.9,0.7,0.1}
\definecolor{jsblue}{rgb}{0,0,0.3}
\definecolor{jslightblue}{rgb}{0,0.3,0.3}

\setbeamercolor{normal text}{fg=black,bg=white}
\setbeamercolor{structure}{fg=orange}
\setbeamercolor*{separation line}{fg=jslightblue}
\setbeamercolor*{fine separation line}{fg=jslightblue}
\setbeamercolor*{frametitle}{fg=orange}

\setbeamerfont{frametitle}{series=\bfseries}

\begin{document}

\title{\Huge{\bf Jenga}\vskip24pt\large\textcolor{black}{Towards a correct and scalable build system}}
\date{
\includegraphics[width=4cm]{new-jane-street-logo.pdf}
}

\addtobeamertemplate{frametitle}{}{%
\begin{textblock*}{3cm}(.77\textwidth,5.4cm)
\vskip26pt\hskip20pt
\includegraphics[width=2cm]{new-jane-street-logo.pdf}
\end{textblock*}\bf}

\setbeamertemplate{navigation symbols}{}

\let\oldframetitle\frametitle
\renewcommand{\frametitle}[1]{%
  \oldframetitle{\vskip4pt\hskip4pt#1}\setstretch{1.4}}

\newcommand\gdb{{\tt gdb}}
\newcommand\gap{\vskip18pt}
\newcommand\highlight[1]{\colorbox{orange}{\textcolor{black}{#1}}}

\begin{frame}
\titlepage
\end{frame}

% This talk is about jenga.
% - a new build tool developed a janestreet - 
% And about the approach taken by jenga to the issues of scalability & correctness.

% To be clear, jenga was very definitely developed with our needs at
% janestreet as priority #1.

% However, jenga is a general purpose build tool & we have made it
% open source, in the hope that it may be of use to other people in
% the ocaml community.
% And so we can showcase some of the non-trading related work which goes on at JS.

% I'm very definitely not here to sell jenga to you!
% But to explain how a complex, real world build process is managed by jenga.


\begin{frame}[fragile]
\frametitle{Why a new build tool?}
\begin{itemize}
\item Already so many to choose from:
\begin{itemize}
\item make, omake, ocamlbuild, ninja, tup, redo, shake...
\end{itemize}
\item Jane Street workflow
\begin{itemize}
\item One big tree! (almost)
\begin{itemize}
\item 3500 dirs; 33k files; 2.2m lines OCaml
\end{itemize}
\item Two workflows
\item Omake: solves many issues, but...
\end{itemize}
\item Can we do better?
\end{itemize}
\end{frame}

% So, the first question is probably: why on earth did we feel the
% need to develop yet another build tool. Does the world not have enough?
%
% To explain, why I need to briefly give some back ground to the way
% our source repositories are handed at JS, and the development
% workflow we use.

% One tree; big - main advantage of a single tree. Can modify
% interfaces to heavily shared common library (improve names/types)
% and fix up all callers in one consistent change-set

% Two flows...
% Continuous integration
% - remote box; full tree; guard bad commits; incremental but restarted
% 
% Individual development
% - local box; own project subtree; incremental, polling
%
% Our build system has to handle both flow.

% Before jenga, janestreet used omake
% - (still uses - we have not finished our switch to jenga)
%
% Omake is pretty good!
% Has lots of features we view as important.
% And solves many issues newer tools fail on.
% Brief list''
% - one build instance for whole tree
% - quick
% - polling
% - dynamic dependencies
% - programmable rule generation
%
% Problems, increasing over time...
% - at JS we have stretched it scalability to the limit
% - more importantly: hard to program
% 
% Thing we dislike the most about omake, is the omake language itself.
% - dynamic typing; dynamic scoping; scripting language not a programming language


\begin{frame}[fragile]
\frametitle{What is a build system?}
\begin{center}
\highlight{Build tool + build rules}
\end{center}
\begin{itemize}
\item The tool is general purpose
\item The rules are provided by the user
\begin{itemize}
\item targets; actions; {\em dependencies}
\item shared framework (e.g. for a whole company)
\item per-project config
\end{itemize}
\end{itemize}
%\item Build process is demand driven
%\item The {\em big} idea...
%\begin{itemize}
%\end{itemize}
%\end{itemize}
\end{frame}

% Tool & rules:
% - Rules provided by user; interpreted by the tool
% - Specialise the build tool for a given project or set-of-projects (at a site)
%
% The build tool is responsible to orchestrate the minimal sequence of
% actions necessary to bring a set of generated targets up to date.

% General purpose tool
% - no build in knowledge of specific languages or compilation tools
% - no magic handling
% - just a consistent general purpose build model

% Rules state what actions must be run to generate which targets.
% And give dependencies for the actions
% - for example: linking stage might describes how some .o are linked into a .exe
% - the .o's are the dependencies.
% - if these are also generated targets (not sources) - likely
% - then before the link is run the .os must be generated, or brought up to date
% - by running whatever action is responsible for creating them
%
% The overall build process is demand driven.
% We request the system to build some top level target, perhaps main.exe


\begin{frame}[fragile]
\frametitle{Rules are hard}
\begin{itemize}
\item Real-world dependencies are difficult
\begin{itemize}
\item Requires detailed knowledge of toolchain
\item Dependencies are not always static
\end{itemize}
\item Our key idea:
\begin{center}
\highlight{Rules are coded in OCaml}
\end{center}
\item Jenga exposes an API
\end{itemize}
\end{frame}

% The most important thing is dependencies.
% Consistent builds REQUIRE accurate dependencies
% Without accurate deps, there can be no hope of a correct incremental builds.
% Developing \& maintaining a build system is a major engineering project.

% Today's talk - scalability & correctness. But focus on correctness.
%
% Real world deps are not trivial, and very often dynamic.
% - dont know what the deps are before the build starts 
% - they are discovered during the build process
% Often we dont even know what targets have rules before we start
% - i.e. setting up a compile rule for all .ml in a directory
%
% We need our build tool to to offer a language rich enough to express
% complex (& dynamic) dependencies for rules and rule generation,
% which exists in the real world 
%
% Early design decision: Rules code in Ocmal, against an Ocaml API.
% - avoiding jet another DSL with crazy syntax and bizarre semantics


\begin{frame}[fragile]
\frametitle{Programming interface}
\begin{itemize}
\item ...for dependencies, rules and rule generation
\item Dynamic dependencies
\begin{itemize}
\item ``scanner dependencies''
\item glob dependencies ({\tt ls *.ml})
\end{itemize}
\item Rule generation
\begin{itemize}
\item not a distinct phase (unlike {\tt ninja})
\item may also have dependencies
\end{itemize}
\item {\tt jengaroot.ml} loaded using {\tt ocaml\_plugin}
\end{itemize}
\end{frame}

% Come to heart of the talk.
% Correctness REQUIRES accurate dependencies
% jenga's approach is to express rules/deps via caml API
% Allow dynamic deps & no sep rule gen phase.

% ocaml plugin
% - type safe dynamic loading of ocaml
% - open sourced
% - code against jenga API
% - caches, so after first time the jengaroot is not recompiled

% Dynamic dependencies are generalisation of what some build tools
% call scanner dependencies. Simplest example: imagine a file which
% contains a list of file names. This file is itself generated.
% A target can be declared to depend on all files listed.
%
% In jenga, arbitrary conditional deps can be expressed
%
% Jenga introduces ``glob dependencies'' i.e. *.ml
% Most commonly used during rule configuration
% And this also works properly in polling mode where we can be
% triggered as new files appear or exiting files disappear

% Finally: rule generation. Not a distinct phase. Integrated with rest
% of build system. For example, rules can be written to have a
% per-directory config file (flags etc) which is consulted for rule-gen.
% This config file could itself be a generated file.


\begin{frame}[fragile]
\frametitle{Rules {\tt make} style}
\begin{itemize}
\item Triple of: targets, dependencies and action
{\footnotesize
\begin{verbatim}
  val rule : path list -> dep list -> act -> rule

\end{verbatim}}
\item Not expressive enough!
\begin{itemize}
\item Can't represent {\em dynamic dependencies}
\item Action is fixed
\end{itemize}
\end{itemize}
\end{frame}

% The concept of a rule in classic make is a static triple.
% Nice any simple.
% Often is all we need
% But in general - it is not expressive enough


\begin{frame}[fragile]
\frametitle{Dependencies}
The fundamental notion is {\it a value and its dependencies}. \par
This is represented by the type:
\begin{center}\colorbox{orange}{$\alpha$\ {\tt dep}}\end{center}
{\footnotesize
\begin{verbatim}

  val depends  : path list -> unit dep
  val contents : path -> string dep
  val glob     : dir:path -> string -> path list dep

\end{verbatim}}
\end{frame}

% So ``dep'' is a parametrised type.
% The simpler dep of the make rules triple, now becomes: unit dep.

% What do values of this type mean?
% Semantically, a value of type {\tt t dep} can be thought of
% as taking different values of type {\tt t} at different times.
%
% These times might be:
% - each time the build tool is restart (we re-evaluate the rules)
% - we re-evaluate as necessary in response to being triggered by external events (inotify)

% explain examples


\begin{frame}[fragile]
\frametitle{Rules {\tt jenga} style}
\begin{itemize}
\item Action is carried by the dependency
{\footnotesize
\begin{verbatim}
  val rule : path list -> act dep -> rule
  
\end{verbatim}}
\item Rule generation
{\footnotesize
\begin{verbatim}
  val generate : (dir:path -> rule list dep) -> scheme

\end{verbatim}}
\item Recover simple rules
{\footnotesize
\begin{verbatim}
  let simple_rule targets deps act =
    rule targets (
      all_unit deps >>= fun () ->
      return act)
\end{verbatim}}
\end{itemize}
\end{frame}

% Action carried by the parametrised dependency type.
% Rule generation fits in the scheme.
% Simple rule type can be discovered.


\begin{frame}[fragile]
\frametitle{The ``{\tt M}'' word}
{\footnotesize
\begin{verbatim}
  val return   : 'a -> 'a dep
  val ( >>= )  : 'a dep -> ('a -> 'b dep) -> 'b dep (* bind *)

  val ( >>| )  : 'a dep -> ('a -> 'b    ) -> 'b dep (* map *)

  val all      : 'a dep list -> 'a list dep
  val all_unit : unit dep list -> unit dep

\end{verbatim}}
\end{frame}

%  Surprise its a monad!
%  This is also the approach takes by shake
%
%  Early version attempted to avoid exposing the monad to the rules
%  author - but this was a mistake.
%
%  Made the rule setup more tricky & much more fragile.


\begin{frame}[fragile]
\frametitle{Example 0: Dependencies}
{\footnotesize
\begin{verbatim}
  val paths_from_file : dir:path -> path -> path list dep

  let paths_from_file ~dir path =
    contents path >>| fun string ->
    let words = split_into_words string in
    List.map words ~f:(fun word -> relative ~dir word)
\end{verbatim}}
\begin{verbatim}
\end{verbatim}
\end{frame}

% Simple example, show monadic style of programming.
% Very familiar to us at JS; rather like coding in the async monad.
%
% Example shows how we read a file using contents, containing a list of names
% and convert this to a list of path w.r.t a give dir, using {\tt relative}
% def of split_into_words is not given
% 
% Another rule would generate the file (e.g. using ocamldep).


\begin{frame}[fragile]
\frametitle{Example 1: Compilation}
{\footnotesize
\begin{verbatim}
  val compile_ml: dir:path -> name:string -> rule
  
  let compile_ml ~dir ~name =
    let p x = relative ~dir (name ^ x) in
    rule [p".cmi"; p".cmx"; p".o"] (
\end{verbatim}
\vskip-16pt\hskip18pt
\colorbox{orange}{\parbox[t][2.3cm][t]{8.7cm}{\ }}\vspace*{-2.5cm}
\begin{verbatim}
      let static = [p".ml"] in
      paths_from_file ~dir (p".ml.d") >>= fun dynamic ->
      depends (static @ dynamic) >>| fun () ->
      bash ~dir (sprintf "ocamlopt -c %s.ml" name)
\end{verbatim}
\begin{verbatim}
    )


\end{verbatim}}
\end{frame}

% All leading up to this big example. I'll take a little time here.
% Sorry about the rather visually disturbing orange rectangle!
%
% This is used to show the 2nd argument to ``rule'', having type ``action dep''
% composed using the map/bind combinators
% Ends in a call to ``bash''
% In this case its static, but it need not be.
% - In the case of linking the bash action would not be static
%
% But this example does how dynamic deps are represented.
% - the call to ``depends'' make use of ``dynamic'', coming from paths_from_file
% - but on RHS of the bind operator.
% - if that file changes, this rule will reconfig itself to be dependant on the new names listed.


\begin{frame}[fragile]
\frametitle{Correctness}
\begin{center}\colorbox{orange}{Incremental =?= Clean}\end{center}
\begin{itemize}
\item Accurate dependencies
\item Single build instance
\begin{itemize}
\item {\em recursive make considered harmful} (P. Miller 1997)
\end{itemize}
\item Digests
\begin{itemize}
\item timestamps are unreliable
\end{itemize}
\item Stale build artifacts
\end{itemize}
\end{frame}

% Single instance
% - Recursive make considered harmful - Peter Miller 1997
% - Build process can only operate correctly with knowledge of the complete dependency graph

% Trying to use timestamps is fraught with difficulties
% e.g. interacts poorly with revision control systems
% And it's just wrong!
% Argument against digest is that they are expensive.
% Weak argument. Cost on restart is avoided by caching
% Using digest can actually avoid rebuild work, by limiting the ripple effect of changes
% - i.e add a comment to an ml.. recompile.. cmx/o are unchanged. no need to relink lib

% Stale build artifacts
% Problem not discussed much in the literature. Except ``tup'' make a deal about it
% A generated file which persists after the rule which created it has been removed
% Problem is can get incremental builds which work.. but from build fails from clean
% - we had found this an issue
% - we had a solution in our omake system + hg, but fits better if build into the build tool

% Most important. Accurate dependencies. But this is the responsibility of the rule author
% (probably the framework author)
% - Again: We need our build tool to to offer a language rich enough to express
% complex (& dynamic) dependencies for rules and rule generation,

% Before move on.
% Tests that our build system is consistent...
%
% (1) Incremental = From-clean
% How many times have you heard - oh, just clean and build again
% clear inditement that the build system is broken
% Not acceptable when a full tree build takes 1--2 hours
%
% (2) Polling = Restart
% not quite in omake - wanted to fix this in jenga


\begin{frame}[fragile]
\frametitle{Scalability}
\begin{itemize}
\item Incremental 
\begin{itemize}
\item {\em the} point of a build system
\end{itemize}
\item Persistence
\begin{itemize}
\item Actions cached by digest
\item Digests cached by {\tt stat}
\end{itemize}
\item Parallel builds
\item Polling (inotify)
\item Per-directory rule generation (unlike {\tt omake})
\end{itemize}
\end{frame}

% These are the features which we want from our build tool which
% relate in some way to scalability.
% Important, but will be brief here, because focus of this talk is jenga's API

% 5 items here. First 4 we had with omake & we retain in jenga.
% jenga adds the last

% If not incremental. Your not a build system, but a script.

% Use persistent store to record record of every action run.
% - digest of deps & targets + the exact command executed.
% so we can tell when we dont have to run again.

% Parallel build: jenga extract maximum parallel as expressed in the rules
% throttled by -j argument
% Commonly choose to set this to number of cores. Or higher and let OS slice.

% After the build has started, and the build tree has been checked & brought up to date
% watches for changes in file using inotify
% only reconsider that part of the build tree effected
% easy to imagine using a different file watching subsystem on different OS
% Goal: quick (enough) to restart, then runs in polling mode...
% Actually, jenga doesn't only start watching files when its done building
% But it watches all the time, even while still building
% And will re-start triggered rebuilds at the earliest moment.
% Call this: tenacious computation (part of core implementation of jenga)

% Per-directory rule gen solves problem in omake, of slow startup, where it has to 
% read & evaluate omakefiles throughout the entire tree, even if we only want to do
% a limited subtree build.

% One nice measure of the speed of a build system is time taken to do
% the null build - where no actions are run


\begin{frame}[fragile]
\frametitle{Summary of Jenga}
\begin{itemize}
\item Key features
\begin{itemize}
\item Rule development in OCaml
\item Expressive API for dynamic dependencies
\item Incremental, parallel, polling builds
\end{itemize}
\item Now builds Jane Street's main tree
\item Open source
\begin{center}
\highlight{\tt opam install jenga}
\end{center}
\vskip12pt
\end{itemize}
\end{frame}

% Rules to build our tree completed (rewrite OMakeroot).
% Used by many developers; not yet CI. (flip soon!)
%
% It's all about dependencies!
%
% If you want to play with jenga, easier way is to get it via use opam.


\begin{frame}[fragile]
\frametitle{More...}
\begin{itemize}
\item API summary
\item Generation example
\end{itemize}
\end{frame}


\begin{frame}[fragile]
\frametitle{API summary}
{\footnotesize
\begin{verbatim}
  val depends  : path list -> unit dep
  val contents : path -> string dep
  val glob     : dir:path -> string -> path list dep
  val relative : dir:path -> string -> path
  val bash     : dir:path -> string -> act
  val rule     : path list -> act dep -> rule
  val generate : (dir:path -> rule list dep) -> scheme
  val ( >>| )  : 'a dep -> ('a -> 'b    ) -> 'b dep (* map *)
  val ( >>= )  : 'a dep -> ('a -> 'b dep) -> 'b dep (* bind *)
  val return   : 'a -> 'a dep
  val all      : 'a dep list -> 'a list dep
  val all_unit : unit dep list -> unit dep

  
\end{verbatim}}
\end{frame}


\begin{frame}[fragile]
\frametitle{Example 2: Rule generation}
{\footnotesize
\begin{verbatim}
  val generate : (dir:path -> rule list dep) -> scheme

  let ocaml_rules : scheme =
    generate (fun ~dir ->
      glob ~dir "*.ml" >>= fun mls ->
      glob ~dir "*.mli" >>| fun mlis ->
      let exists_mli x = List.mem mlis (relative ~dir (x ^ ".mli")) in
      List.concat_map mls ~f:(fun ml ->
        let name = chop_suffix (basename ml) ".ml" in
        if (exists_mli name)
        then compile_ml_mli ~dir ~name
        else compile_ml ~dir ~name))


\end{verbatim}}
\end{frame}

\end{document}
